\documentclass[a4paper, 14pt]{article}
\usepackage[T2A]{fontenc}
\usepackage[utf8]{inputenc}
\usepackage[english,russian]{babel}
\usepackage[top = 2cm, bottom = 2 cm]{geometry}
\usepackage{cmap}
\usepackage{graphicx}
\usepackage{listings}
\usepackage{color}
\usepackage{amsmath}
\usepackage{pgfplots}
\usepackage{url}
\usepackage{tikz}
\usepackage{float}
\usepackage{multirow}
\usepackage{indentfirst}
\usepackage{longtable}
\usepackage{array}
\usepackage{titlesec}
\titleformat*{\section}{\LARGE\bfseries}
\titleformat*{\subsection}{\Large\bfseries}
\titleformat*{\subsubsection}{\large\bfseries}
\titleformat*{\paragraph}{\large\bfseries}
\titleformat*{\subparagraph}{\large\bfseries}


\begin{document}

\section*{Задание}

Вариант 8 

\begin{enumerate}

\item Назначить адреса подсетей
\begin{itemize}
\item	Подсеть 1: 192.168.8.0 /24
\item	Подсеть 2: 192.168.9.0 /24
\item	Подсеть 3: 192.168.10.0 /24
\item	Подсеть 4: 192.168.11.0 /24
\item	Подсеть 5 (В задаче 3): 192.168.18.0 /24
\end{itemize}

\item Настроить динамическую маршрутизацию в прилагаемом .pkt файле на стенде I через протокол RIPv2 так, чтобы пинг любым хостом или маршрутизатором любого другого хоста или маршрутизатора был успешным.
Представить отдельным .pkt файлом. 

\item Настроить динамическую маршрутизацию в сети в прилагаемом .pkt файле на стенде II через протокол OSPF так, чтобы пинг любым хостом или маршрутизатором любого другого хоста или маршрутизатора был успешным. Разделить при этом сеть на области OSPF в соответствии со схемой. Выполнить указания в лабораторной работе.
Представить отдельным .pkt файлом. 


\end{enumerate}

\section*{Разделение на подсети}

\begin{table}[H]
    \centering
    \caption{Разделение на подсети на стенде I}
    \begin{tabular}{|p{0.8cm}|p{3cm}|p{6cm}|}
        \hline
        № подсети & IP-адрес подсети & Диапазон адресов  \\
        \hline
        1 & 192.168.8.0 & 192.168.8.1 - 192.168.8.2  \\
        \hline
        2 & 192.168.7.0 & 192.168.9.1 - 192.168.9.2  \\
        \hline
        3 & 192.168.10.0 & 192.168.10.1 - 192.168.10.2  \\
        \hline
        4 & 192.168.11.0 & 192.168.11.1 - 192.168.11.2 \\
        \hline
    \end{tabular}
\end{table}

\begin{figure}[H]
    \includegraphics[scale=0.5]{stend1}
    \caption{Разделение на подсети на стенде I}
\end{figure}

\newpage
\begin{table}[H]
    \centering
    \caption{Разделение на подсети на стенде II}
    \begin{tabular}{|p{0.8cm}|p{3cm}|p{6cm}|}
        \hline
        № подсети & IP-адрес подсети & Диапазон адресов  \\
        \hline
        1 & 192.168.8.0 & 192.168.8.1 - 192.168.8.2  \\
        \hline
        2 & 192.168.7.0 & 192.168.9.1 - 192.168.9.2  \\
        \hline
        3 & 192.168.10.0 & 192.168.10.1 - 192.168.10.2  \\
        \hline
        4 & 192.168.11.0 & 192.168.11.1 - 192.168.11.2 \\
        \hline
        5 & 192.168.18.0 & 192.168.18.4 - 192.168.11.2 \\
        \hline
    \end{tabular}
\end{table}

\begin{figure}[H]
    \includegraphics[scale=0.6]{stend2}
    \caption{Разделение на подсети на стенде II}
\end{figure}


\newpage
\section*{Настройка RIPv2}

Для корректной работы динамической маршрутизации необходимо настроить все роутеры для использования RIPv2.\\

\textbf{network network\_num}, где network\_num - адрес сети. Позволяет добавить сеть/диапазон адресов, который будет использоваться для рассылки обновлений RIP.

\textbf{version 2} - изменение версии RIP на RIPv2\\


На рис. \ref{fig:1} показана настройка RIPv2 для Router0. Остальные роутеры настраиваются аналогично.

\begin{figure}[H]
	\begin{center} 
    \includegraphics[scale=0.5]{router0}
    \caption{CLI роутера Router0}
    \label{fig:1}
    \end{center}
\end{figure}

Пинг PC3 из PC0:

\begin{figure}[H]
	\begin{center} 
    \includegraphics[scale=0.5]{ping1}
    \caption{Пинг PC3 из PC0}
    \label{fig:2}
    \end{center}
\end{figure}

\newpage
\section*{Настройка OSPF}

Все роутеры были настроены для динамической маршрутизации через протокол OSPF.

На рис. \ref{fig:3} -  \ref{fig:6} показана настройка каждого роутера.

\begin{figure}[H]
	\begin{center} 
    \includegraphics[scale=0.5]{router7}
    \caption{CLI роутера Router7}
    \label{fig:3}
    \end{center}
\end{figure}

\begin{figure}[H]
	\begin{center} 
    \includegraphics[scale=0.5]{router8}
    \caption{CLI роутера Router8}
    \label{fig:4}
    \end{center}
\end{figure}

\begin{figure}[H]
	\begin{center} 
    \includegraphics[scale=0.5]{router9}
    \caption{CLI роутера Router9}
    \label{fig:5}
    \end{center}
\end{figure}

\begin{figure}[H]
	\begin{center} 
    \includegraphics[scale=0.5]{router10}
    \caption{CLI роутера Router10}
    \label{fig:6}
    \end{center}
\end{figure}

\newpage

Результат выполнения команды \textbf{sh ip ospf neighbor} для роутера Router8:


\begin{figure}[H]
	\begin{center} 
    \includegraphics[scale=0.5]{neighbor}
    \caption{CLI роутера Router8}
    \label{fig:7}
    \end{center}
\end{figure}


\textbf{DR} -  Router10\\

\textbf{BDR} -  Router9\\

Роль \textbf{ABR} имеют все роутеры, т.к. все они соединены с различными зонами.\\

Пинг PC9 из PC8:

\begin{figure}[H]
	\begin{center} 
    \includegraphics[scale=0.5]{ping2}
    \caption{Пинг PC9 из PC8}
    \label{fig:2}
    \end{center}
\end{figure}

\end{document}