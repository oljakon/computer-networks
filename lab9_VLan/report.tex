\documentclass[a4paper, 14pt]{article}
\usepackage[T2A]{fontenc}
\usepackage[utf8]{inputenc}
\usepackage[english,russian]{babel}
\usepackage[top = 2cm, bottom = 2 cm]{geometry}
\usepackage{cmap}
\usepackage{graphicx}
\usepackage{listings}
\usepackage{color}
\usepackage{amsmath}
\usepackage{pgfplots}
\usepackage{url}
\usepackage{tikz}
\usepackage{float}
\usepackage{multirow}
\usepackage{indentfirst}
\usepackage{longtable}
\usepackage{array}
\usepackage{titlesec}
\titleformat*{\section}{\LARGE\bfseries}
\titleformat*{\subsection}{\Large\bfseries}
\titleformat*{\subsubsection}{\large\bfseries}
\titleformat*{\paragraph}{\large\bfseries}
\titleformat*{\subparagraph}{\large\bfseries}

\begin{document}

\section*{Задание}

Вариант 8 

\begin{enumerate}

\item Назначить адреса подсетей
\begin{itemize}
\item	Подсеть 1: 192.168.8.0 /24
\item Подсеть 2: 192.168.9.0 /24
\item	Подсеть 3: 192.168.10.0 /24
\end{itemize}

\item Настроить поддержку трех виртуальных локальных сетей (VLan 10, 20, 30) на коммутаторе.

\item Настроить маршрутизацию между виртуальными локальными сетями на маршрутизаторе.

\item Выделить и озаглавить на схеме каждую виртуальную локальную сеть.

\end{enumerate}

\section*{Разделение на подсети}

\begin{table}[H]
    \centering
    \caption{Разделение на подсети на стенде II}
    \begin{tabular}{|p{0.8cm}|p{3cm}|p{6cm}|}
        \hline
        № подсети & IP-адрес подсети & Диапазон адресов  \\
        \hline
        1 & 192.168.8.0 & 192.168.8.1 - 192.168.8.254  \\
        \hline
        2 & 192.168.9.0 & 192.168.9.1 - 192.168.9.254  \\
        \hline
        3 & 192.168.10.0 & 192.168.10.1 - 192.168.10.254  \\
        \hline
    \end{tabular}
\end{table}

\begin{figure}[H]
    \includegraphics[scale=0.7]{subnet}
    \label{fig:1}
\end{figure}

\section*{Настройка поддержки виртуальных локальных сетей}

Была настроена поддержка трех виртуальных локальных сетей на коммутаторе. Ниже приведены команды, которые вводились в CLI коммутатора.

\begin{figure}[H]
    \includegraphics[scale=0.5]{1}
    \label{fig:1}
\end{figure}
\begin{figure}[H]
    \includegraphics[scale=0.5]{2}
    \label{fig:1}
\end{figure}
\begin{figure}[H]
    \includegraphics[scale=0.5]{3}
    \label{fig:1}
\end{figure}
\begin{figure}[H]
    \includegraphics[scale=0.5]{4}
    \label{fig:1}
\end{figure}

\newpage
Результат можно проверить с помощью команды \textbf{show vlan}:

\begin{figure}[H]
    \includegraphics[scale=0.5]{show}
    \label{fig:1}
\end{figure}

Далее приведены выставленные в результате выполнения команд настройки у интерфейсов в поле VLan.

\begin{figure}[H]
    \includegraphics[scale=0.5]{5}
    \label{fig:2}
\end{figure}
\begin{figure}[H]
    \includegraphics[scale=0.5]{6}
    \label{fig:3}
\end{figure}
\begin{figure}[H]
    \includegraphics[scale=0.5]{7}
    \label{fig:4}
\end{figure}
\begin{figure}[H]
    \includegraphics[scale=0.5]{8}
    \label{fig:5}
\end{figure}

\newpage
\section*{Настройка маршрутизации между виртуальными локальными сетями на маршрутизаторе}

Для настройки маршрутизатора были введены следующие команды: 

\begin{figure}[H]
    \includegraphics[scale=0.5]{9}
    \label{fig:6}
\end{figure}
\begin{figure}[H]
    \includegraphics[scale=0.5]{10}
    \label{fig:6}
\end{figure}
\begin{figure}[H]
    \includegraphics[scale=0.5]{11}
    \label{fig:6}
\end{figure}
\begin{figure}[H]
    \includegraphics[scale=0.5]{12}
    \label{fig:6}
\end{figure}

После в конечных узлах всех трех подсетей в качестве шлюза по умолчанию были выставлены адреса, приведенные выше.

\newpageИзучение технологии виртуальных локальных сетей (VLan) в сетевом симуляторе. Настройка маршрутизации между VLan.
\section*{Виртуальные локальные сети}

На рисунке ниже выделена каждая виртуальная локальная сеть.

\begin{figure}[H]
    \includegraphics[scale=0.7]{vlan}
    \label{fig:7}
\end{figure}

Результат проверки соединения мужду Server0 и PC3:

\begin{figure}[H]
    \includegraphics[scale=0.7]{ping}
    \label{fig:8}
\end{figure}

\end{document}